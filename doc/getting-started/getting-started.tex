%%%%%%%%%%%%%%%%%%%%%%%%%%%%%%%%%%%%%%%%%%%%%%%%%%%%%%%%%%%%%%%%%%%%%%%%%%%%%%%
%  © Université de Lille, The Pip Development Team (2015-2024)                %
%  Copyright (C) 2020-2024 Orange                                             %
%                                                                             %
%  This software is a computer program whose purpose is to run a minimal,     %
%  hypervisor relying on proven properties such as memory isolation.          %
%                                                                             %
%  This software is governed by the CeCILL license under French law and       %
%  abiding by the rules of distribution of free software.  You can  use,      %
%  modify and/ or redistribute the software under the terms of the CeCILL     %
%  license as circulated by CEA, CNRS and INRIA at the following URL          %
%  "http://www.cecill.info".                                                  %
%                                                                             %
%  As a counterpart to the access to the source code and  rights to copy,     %
%  modify and redistribute granted by the license, users are provided only    %
%  with a limited warranty  and the software's author,  the holder of the     %
%  economic rights,  and the successive licensors  have only  limited         %
%  liability.                                                                 %
%                                                                             %
%  In this respect, the user's attention is drawn to the risks associated     %
%  with loading,  using,  modifying and/or developing or reproducing the      %
%  software by the user in light of its specific status of free software,     %
%  that may mean  that it is complicated to manipulate,  and  that  also      %
%  therefore means  that it is reserved for developers  and  experienced      %
%  professionals having in-depth computer knowledge. Users are therefore      %
%  encouraged to load and test the software's suitability as regards their    %
%  requirements in conditions enabling the security of their systems and/or   %
%  data to be ensured and,  more generally, to use and operate it in the      %
%  same conditions as regards security.                                       %
%                                                                             %
%  The fact that you are presently reading this means that you have had       %
%  knowledge of the CeCILL license and that you accept its terms.             %
%%%%%%%%%%%%%%%%%%%%%%%%%%%%%%%%%%%%%%%%%%%%%%%%%%%%%%%%%%%%%%%%%%%%%%%%%%%%%%%

\documentclass[10pt,a4paper,titlepage]{refart}
\usepackage[utf8x]{inputenc}
\usepackage{ucs}
\usepackage[english]{babel}
\usepackage{amsmath}
\usepackage{amsfonts}
\usepackage{amssymb}
\usepackage{listings}
\usepackage{xcolor}
\usepackage{hyperref}
\usepackage{cleveref}
\usepackage{comment}
\usepackage{multirow}

\lstset{columns=fullflexible}

\title{Get Started with Pip}

\definecolor{commentcolor}{rgb}{0,0.6,0}
\definecolor{numbercolor}{rgb}{0.5,0.5,0.5}
\definecolor{stringcolor}{rgb}{0.58,0,0.82}
\definecolor{backgroundcolour}{rgb}{0.95,0.95,0.92}

\lstset {
    backgroundcolor=\color{backgroundcolour},
    basicstyle=\footnotesize,
    breakatwhitespace=false,
    breaklines=true,
    keepspaces=true,
    showspaces=false,
    showstringspaces=false,
    showtabs=false,
    tabsize=4,
    frame=single
}

\iffalse
\lstset{
basicstyle=\ttfamily,
frame=single,
breaklines=true,
showstringspaces=false,
}
\fi

\lstdefinestyle{BashStyle} {
    commentstyle=\color{commentcolor},
    keywordstyle=\color{black},
    stringstyle=\color{stringcolor},
    language=bash
}

\lstdefinestyle{CStyle} {
    emph={uint32_t},
    emphstyle={\color{magenta}},
    commentstyle=\color{commentcolor},
    keywordstyle=\color{magenta},
    stringstyle=\color{stringcolor},
    language=C
}

\hypersetup{
    colorlinks,
    citecolor=black,
    filecolor=black,
    linkcolor=black,
    urlcolor=black
}

\author{Nicolas Dejon - Damien Amara - Claire Soyez-Martin}
\title{Pip-MPU 101 -  Tutorial \\ Getting started on ARM Cortex-M \\ Example with a simple Hello world and a scheduler partition on the DWM1001-DEV (Arm Cortex-M4) board\\}

\begin{document}
\maketitle
\tableofcontents
%\addcontentsline{toc}{section}{Listings}
%\lstlistoflistings
\pagebreak

This tutorial sets up a building environment for Pip-MPU and gets you through the launch of the \href{(https://gitlab.univ-lille.fr/2xs/pip/pip-mpu-scheduler)}{scheduler partition example}.

It assumes Ubuntu 22.04 LTS users for the commands, but Pip-MPU is known to build on other Linux distributions as well using the appropriate package manager.

The tutorial targets the \href{https://www.qorvo.com/products/d/da007952}{DWM1001-DEV board} by Qorvo which relies on the \href{https://www.nordicsemi.com/products/nrf52832}{nRF52832 chip} (ARM Cortex-M4) by Nordic Semiconductor.

The tutorial is split into three parts.
The first part, \cref{first}, sets up a working build environment and builds the Pip-MPU kernel.
The second part, \cref{second}, builds a simple Hello world program and links it with the kernel in an executable file.
The third part, \cref{third}, explains how to build a partition, using the example of a scheduler.
The fourth part, \cref{fourth}, addresses the flash and run of the executable on a real, local or remote, board.

\section{Background}

\subsection{Kernel structure}

The kernel is divided into three parts.

\begin{itemize}
    \item HAL: The Hardware Abstraction Layer is used to provide small functions to manipulate the MPU
    \item Core: The logic of Pip-MPU (kernel services)
    \item Boot: The bootstrap code that initializes required hardware and then boots Pip-MPU
\end{itemize}

\subsection{Executable workflow}
The Pip-MPU kernel and its partitions are compiled separately.

Concerning Pip-MPU, Pip-MPU's services (Core folder) are written in Gallina, which is automatically translated in the C language by the dx tool 
(or, in more ancient commits, by the Digger tool). HAL and Boot are written in C/ASM.
Pip-MPU's \texttt{Makefile} includes Digger's translation into C, before compiling the whole as any C project.

Concerning the partitions, they are written in C/ASM.
They have their own \texttt{Makefile} and compile as any other project.

Hence, in order to get an executable file, a script links together the Pip-MPU kernel and the partitions compiled separately to form a unique executable.
You will get through all these steps in the following.

\section{Part I: Build the kernel} \label{first}

\subsection{Build environment}
In order to create your own partitions on top of Pip-MPU, you need an appropriate development environment, relying on several tools.
These tools are needed for cross-compiling bare-metal C for ARM Cortex-M chips.

\index{gcc}\marginlabel{GNU C Cross Compiler:}
arm-none-eabi-gcc is the only C compiler known to compile Pip-MPU correctly.
CLANG, for example, is not supported. To that end, you need a version of GCC capable of producing 32bits ELF binaries.

The last working version is v12.1.0.

\index{gdb-multiarch} \marginlabel{GNU Debugger multiarch:}
The GNU Debugger allows you to debug a partition while it is executed on the
top of Pip. This is very useful during the development process. That's the
reason why GDB is not mandatory but highly recommended.


\index{stack}\marginlabel{Haskell Stack:}
In old versions of the project, Pip-MPU used Digger, a home-made extractor to convert Coq code into C code.
In order to compile this Extractor, which is written in Haskell, we use the Stack tool to download and install automatically the required GHC and libraries.

The last working version of Stack is v2.7.5 and Digger's is v0.3.0.0.

\index{coq}\marginlabel{Coq Proof Assistant:}
Pip-MPU's source code and formal proof of its memory isolation properties are written
using the Coq proof assistant. In order to compile Coq files and
generate the required intermediate files for the kernel to build, you will need the 8.13.1 version of Coq.

This was also tested successfully with Coq 8.15.0.

\index{opam} \marginlabel{OCaml Package Manager:}
Opam is the package manager for the OCaml programming language, the language in
which Coq is implemented. This is the proper way to install and pin the Coq
Proof Assistant to a specific version.

\index{gnu}\marginlabel{GNU Toolchain:}
GNU Sed and GNU Make are required to perform the compilation.

The last working versions of sed is v4.8 and of make is v4.3.

\index{doxygen}\marginlabel{Doxygen:}
Pip-MPU's documentation is generated through CoqDoc (included with Coq) for the Coq part, and Doxygen for the C part.
The documentation is not mandatory to compile Pip-MPU, but it is highly required that you compile it and keep it somewhere safe so you always have some reference to read if you need some information about Pip-MPU's internals.

The last working version of doxygen is v1.9.3.

\subsection{Install required packages}

Update the apt package index before installing the packages:

\begin{lstlisting}[style=BashStyle]
> sudo apt update
> sudo apt install git make opam gdb-multiarch texlive-latex-base doxygen haskell-stack
\end{lstlisting}

In addition to these packages, it is necessary to install the ARM GNU toolchain. As of today, it is preferable to to so manually\footnote{See the first 
answer in \url{https://askubuntu.com/questions/1243252/how-to-install-arm-none-eabi-gdb-on-ubuntu-20-04-lts-focal-fossa}}, by downloading the latest version
on \url{https://developer.arm.com/downloads/-/arm-gnu-toolchain-downloads} and creating the links to make the binaries accesible system-wide. On an Ubuntu system, 
these links are created as follows:
\begin{lstlisting}[style=BashStyle]
> sudo ln -s /usr/share/gcc-arm-none-eabi-YOUR-VERSION/bin/arm-none-eabi-gcc /usr/bin/arm-none-eabi-gcc 
> sudo ln -s /usr/share/gcc-arm-none-eabi-YOUR-VERSION/bin/arm-none-eabi-g++ /usr/bin/arm-none-eabi-g++
> sudo ln -s /usr/share/gcc-arm-none-eabi-YOUR-VERSION/bin/arm-none-eabi-gdb /usr/bin/arm-none-eabi-gdb
> sudo ln -s /usr/share/gcc-arm-none-eabi-YOUR-VERSION/bin/arm-none-eabi-ld /usr/bin/arm-none-eabi-ld
> sudo ln -s /usr/share/gcc-arm-none-eabi-YOUR-VERSION/bin/arm-none-eabi-size /usr/bin/arm-none-eabi-size
> sudo ln -s /usr/share/gcc-arm-none-eabi-YOUR-VERSION/bin/arm-none-eabi-objcopy /usr/bin/arm-none-eabi-objcopy
\end{lstlisting}

Note that only some of these tools (arm-none-eabi-gcc and arm-none-eabi-ld) are necessary for this project.

\subsection{Get the source}
\index{Pipsrc}\marginlabel{Pip-MPU source:}
Once your development environment is ready, open a terminal and clone the Pip-MPU repository from \texttt{https://github.com/2xs/pipcore-mpu}.
\textit{Warning:} replace by this repository's name.

\begin{lstlisting}[style=BashStyle]
> git clone https://gitlab.univ-lille.fr/2xs/pip/pipcore-mpu.git
> cd pipcore-mpu
\end{lstlisting}

Pip-MPU's last version should be compatible with the two example projects. If a problem arises, try this version:

\begin{lstlisting}[style=BashStyle]
> git checkout 2472b956494caa20b9932e8348c2310a949440ff
\end{lstlisting}

This repository contains Pip-MPU's kernel, proof and documentation. Still,
it is not ready to compile yet, as it provides no partition to run.
This is covered in \cref{second}.

% Opam setup is required to initialize dx in the further steps.
% It might be relevant to guide the user through opam installation first. 

\index{coqcsrc}\marginlabel{Set up the Coq 8.13.1 environment:}
We use opam to install Coq tools required to extract the code base.

\begin{lstlisting}[style=BashStyle]
> opam switch create coq8.13.1 4.10.2
> opam init
> eval $(opam env)
# pin Coq version to 8.13.1, these could take a few minutes
> opam pin add coq 8.13.1
# test installation, the command must end successfully
> coqc -v
\end{lstlisting}

\index{diggersrc}\marginlabel{Set up Digger:}
As said previously, old versions of Pip-MPU used an extractor to convert Coq code into C code.
For the sake of completion, here is how to get it: the source is included in Pip-MPU's main repository as a submodule.
\begin{lstlisting}[style=BashStyle]
> pipcore-mpu$ git submodule init
# checkout to expected version
> pipcore-mpu$ git submodule update
# build Digger, these could take a few minutes
> pipcore-mpu$ make -C tools/digger
\end{lstlisting}
The compilation might complain about a missing GHC version : type in the asked command to install the required GHC version (usually "\texttt{stack setup}"), and type in \texttt{make} again to grab the required libraries and build the extractor.

\index{dxsrc}\marginlabel{Set up dx:}
The most recent version of pipcore-mpu uses dx instead of Digger to get the C code. The README 
of that project details the necessary steps to compile it, but here is an abridged version.

\begin{lstlisting}[style=BashStyle]
> git clone https://gitlab.univ-lille.fr/2xs/dx
> cd dx
> opam repository add coq-released https://coq.inria.fr/opam/released
> opam install -b coq coq-elpi coq-compcert
> opam pin .
\end{lstlisting}

Once both pipcore-mpu and dx are compiled, it is possible to use dx in the execution of pipcore-mpu by adding an optionnal argument, as explained below.

\subsection{Build Pip-MPU} \label{buildpip}
Before doing anything else, you need to configure the partition building toolchain for the dwm1001 board.
The optional argument \textit{--debugging-mode=semihosting} enables semihosting outputs.
Then, compile the kernel.
% TODO: remove --no-command-check
Note that the partitions use the same building toolchain as Pip-MPU's.

To compile the kernel, the path to the dx library is required. That path can change depending on how dx was installed. 
If it was installed using opam, the path should be of the form

\begin{lstlisting}[style=BashStyle]
/home/<user>/.opam/coq8.13.1/lib/dx
\end{lstlisting}

where $<$user$>$ is the username used on your computer. Then, using this path, it is possible to compile the kernel with 
the following commands:

\begin{lstlisting}[style=BashStyle]
# configure the building toolchain
> pipcore-mpu$ ./configure.sh --architecture=dwm1001 --debugging-mode=semihosting --no-command-check  --boot-sequence=default --dx=<path to dx>
# compile the kernel
> pipcore-mpu$ make all
# check Pip-MPU has successfully compiled:
> pipcore-mpu$ ls | grep pip.bin
\end{lstlisting}

If the last instruction returns \textit{pip.bin} as output, congratulations, you successfully compiled Pip-MPU!

\section{Part II: Link the kernel to another program with the example of a Hello world} \label{second}

Now that Pip-MPU is built, we can link it to the executable file of another project. For that, it is necessary to compile that project 
separately and then package it with the kernel in order to get the final binary. In order to illustrate this part, we consider a simple 
Hello world program, that can be found \href{(https://gitlab.univ-lille.fr/2xs/pip/pip-mpu-hello-world)}{here}.

\subsection{Download the source}
Clone the example in another directory \textit{"hello world dir"}, outside the Pip-MPU kernel.

\begin{lstlisting}[style=BashStyle]
> cd <another directory>
> git clone https://gitlab.univ-lille.fr/2xs/pip/pip-mpu-hello-world.git
> cd pip-mpu-hello-world
\end{lstlisting}

\subsection{Compile the project}
This project inherits Pip-MPU's toolchain configuration (set up at \ref{buildpip}).
It is then enough to invoke the \texttt{make} command.
Note that it will require a matching version of elftools. Install it using pip install -r.

\begin{lstlisting}[style=BashStyle]
> pip-mpu-hello-world$ pip install -r relocator/requirements.txt
> pip-mpu-hello-world$ make
\end{lstlisting}

\subsection{Link the project with the kernel}
Return to the kernel's directory and link Pip-MPU's binary with the project's binary. To do so, it suffices to concatenate 
them as follows:

\begin{lstlisting}[style=BashStyle]
> pipcore-mpu$ cat pip.bin <another directory>/pip-mpu-hello-world/pip-mpu-hello-world.bin > pip+root.bin
\end{lstlisting}

The script produces the final binary, that can then be flashed and run, as shown in \cref{fourth}.

\subsection{Structure}\label{subsec:structure}
The binaries of the project and of Pip-MPU's kernel have been concatenated, so that the project runs only after Pip-MPU is 
initialized. That initialization sets up the memory so that Pip-MPU has a reserved memory block isolated from the rest of the system, 
meaning that nothing in the system can access that memory. Moreover, Pip-MPU's initialization retrieves information about that memory 
block, notably the exact physical addresses included in it, and creates a root partition, that is to be initialized in another project 
(in our case, the hello-word project). This creates some information about that root partition, that is passed down to the other 
project using an interface.

The hello-world project is compiled using a file named \textit{link.ld}, which indicates in which order the different parts should be 
put in the final executable file. Notably, it indicates that the first function to be executed after Pip-MPU's initialization is the 
\textit{\_start} function from the file \textit{crt0.c}. That function stores the necessary data about the available memory, then calls 
the \textit{main} function from the file \textit{main.c}, to which it gives as parameter the interface it received from Pip-MPU. That 
interface gives information about the root partition initialized by Pip-MPU, notably the ID of its partition descriptor, which is used 
in most of Pip-MPU's primitives. %TODO detail that?

In this project, the \textit{main} function does not look at any parameter it might receive nor returns anything. The fact that it does 
not look at its parameters is specific to simple programs, but the absence of a return value characterizes the programs handled by Pip-MPU. 
Indeed, this kind of program is supposed to be launched then to run indefinitely, needing no external interventions to continue if there 
are no errors. Thus, the  \textit{main} function first calls the auxilliary function \textit{puts} to print the string "Hello World!" and 
begin a new line, then it loops forever.

\begin{lstlisting}[language=C]
void main(void)
{
    puts("Hello World!\n");
    for (;;);
}
\end{lstlisting}

The \textit{puts} function prints a string, and uses the assembly language to communicate with the hosting system.
%TODO detail that?

\section{Part III: Tutorial on partitions} \label{third}

This part is intended to give a more complete tutorial on Pip, by explaining how to construct the partition presented in the 
\href{(https://gitlab.univ-lille.fr/2xs/pip/pip-mpu-scheduler)}{partition example}. This project can be compiled exactly as 
the one presented in \cref{second}, and linked to Pip-MPU, as it is a partition, an thus inherits Pip-MPU's toolchain configuration.
The idea of the project is to create a child partition and to alternate between it and the root partition to print messages, all 
using Pip's interface. Here, we will see how to use this interface to create the project.

\subsection{Download and compile the project}

\marginlabel{Download the source}

Clone the project in another directory, outside the Pip-MPU kernel.

\begin{lstlisting}[style=BashStyle]
> cd <another directory>
> git clone https://gitlab.univ-lille.fr/2xs/pip/pip-mpu-launcher.git
> cd pip-mpu-launcher
\end{lstlisting}

\marginlabel{Compilation}

This project inherits Pip-MPU's toolchain configuration (set up at \ref{buildpip}). It is then enough to invoke the \texttt{make} 
command.

\begin{lstlisting}[style=BashStyle]
> pip-mpu-launcher$ make
\end{lstlisting}

This command produces a binary file pip-mpu-launcher{.}bin, which can ben used to produce the final binary with a call to cat:

\begin{lstlisting}[style=BashStyle]
> pip-mpu-launcher$ cat <Pip-MPU directory>/pip.bin pip-mpu-launcher.bin > pip+root.bin
\end{lstlisting}

That final binary can then be flashed on the target board (see \cref{fourth}).

\subsection{Overview}\label{subsec:overview}

This subsection is intended to give a general overview of how to write a program dealing with partitions using Pip. It focuses on 
the most important primitives, leaving aside the parts that deal with memory blocks.

\subsubsection{Structure}

There are several ways to write a project compatible with Pip. The one presented here is hopefully the simplest to understand, but 
not the most efficient, so feel free to improve it. The project presented here is structured with three folders:
\begin{itemize}
    \item include contains the libraries necessary for the compilation,
    \item child contains the files necessary for the compilation of the child partition,
    \item root contains the files necessary for the compilation of the root partition.
\end{itemize}

In this project, the "include" folder contains only one file, that enables semihosting and that will not be detailed here.

Both the "child" and "root" folders have the same structure: a "relocator" folder whose files deal with the structure of the memory, 
a "link{.}ld" file that deals with the structure of the final binary file, a "main{.}c" file containing the partition's program, and 
a Makefile that defines the command \texttt{make} for that folder. This structure allows each partition to be separately compiled, 
even though the goal is to combine the binaries. In what follows, we focus on the contents of the "main{.}c" files, as the others 
deal with memory alignments and contain a lot of technical details.

\subsubsection{The child partition}

\marginlabel{The code}

Before talking about how to deal with partitions, let's talk about what we want our child partition to do. That partition is rather 
simple, as it does not have to deal with memory: its memory has already been allocated and initialized by the root partition, and 
it has no child partition. Consequently, all it has to do is to execute its own program and call back the root partition at least 
once, when it is done. In this example project, we want to show how Pip's primitives work, so the child partition will call the 
root several times. The goal is to proceed as follows:
\begin{itemize}
    \item print "3" on the screen, then call back the parent (in our case, the root),
    \item if the child is called again, print "2" on the screen, then call back the parent,
    \item if the child is called a third time, print "1" on the screen, then call back the parent,
    \item if the child is called more than three times, loop indefinitely without doing anything.
\end{itemize}
The entry point of the child partition is the function \textit{start} of the file child/main{.}c, defined as follows:
\begin{lstlisting}[language=C]
void start(void)
{
    ...

	for (;;);
}
\end{lstlisting}
Note the infinite loop, which is the last instruction of the function start. A program on an MPU is not supposed to stop unless an error 
occurs, and this is how we emulate it.

The rest of the function is dealt with by two functions. To print strings, we use the function \textit{puts}, defined in the file 
include/semihosting{.}h, which takes a constant string of characters and prints it. The change of partition is done through the function 
\textit{Pip\_yield}, which takes the ID of the partition to switch to and information about how to store and restore the contexts, and 
yields to the indicated partition, restoring the indicated context with the indicated flags. That function uses particular tables, called 
VIDTs. Each partition has exactly one VIDT, in which it stores contexts of execution. Each index of a VIDT corresponds to an interruption, 
or to a way of changing the partition. In this project, we do not use interruptions, so we will always store and retrieve the contexts at 
the index 0 of the VIDTs. Formally, the function Pip\_yield takes five parameters:
\begin{itemize}
    \item a pointer calleePartDescBlockId, which indicates the ID of the memory block containing the partition description of the 
    partition we want to yield to (a value of 0 calls the parent of the current partition, which is the partition itself when it is the 
    root),
    \item an integer userTargetInterrupt, which gives the index of the callee's VIDT where to retrieve the context of execution needed by 
    that partition,
    \item an integer userCallerContextSaveIndex, which gives the index of the caller's VIDT where to store the current context of 
    execution,
    \item an integer flagsOnYield, which sets the interruption flags during the yield operation (not important in ou case, so we always 
    leave it at 1),
    \item an integer flagsOnWake, which sets the interruption flags when the partition is reawakened (not important in ou case, so we always 
    leave it at 1).
\end{itemize}
Consequently, to print "3" then call back its parent, the child must use the following code:
\begin{lstlisting}[language=C]
puts("3\n");
Pip_yield(0, 0, 0, 1, 1);
\end{lstlisting}
In the call to Pip\_yield, we call the parent by setting the first parameter to 0, we store the current context at the index 0 of the 
child's VIDT, and we retrive the parent's context at the index 0 of its VIDT.

Since we want to repeat that three times, we repeat these two lines, changing only the string in the call to puts. Each time Pip\_yield is 
called, the current context of execution is stored so that, if the parent calls back the child partition, the execution is resumed at the 
line where it yielded to the parent, causing the child to print "3, then yield, then print "2" if it is called back, instead of printing "3" 
again.

\marginlabel{Separate execution}

The child program is supposed to be called by a parent disjoint from it, but it can be executed as the root partition. In that case, the 
calls to Pip\_yield will store the child's context then immediately restore it. Thus, the program would print "3", then "2", then "1", and 
loop indefinitely, doing nothing more. %TODO say here how to compile the project?

\subsubsection{Dealing with partitions: the root partition}

The goal of this tutorial is to show how to deal with partitions using Pip, so we define a root partition that will create a partition for 
the child defined above, yields three times, then deletes the partition to clean the memory, and loops indefinitely. The entry point for 
that root partition is the function \textit{start} of the file root/main{.}c, defined as follows:
\begin{lstlisting}[language=C]
void start(interface_t *interface)
{
    ...

    for (;;);
}
\end{lstlisting}
Note that, contrary to the child's entry point, this function does not ignore its parameters, instead taking a pointer to a structure. That 
structure contains all the necessary information about the memory to correctly set the partitions. However, here we will not detail how to 
use it.

The code for the root partition can be separated into five parts, each having a distinct goal. Four of them are handled by functions 
defined in the "main{.}c" file:
\begin{itemize}
    \item \textit{root\_memory\_init} sets up the memory blocks necessary for both the root and the child partitions, and uses the 
    interface given by Pip-MPU to do so,
    \item \textit{root\_partition\_create} creates the child partition and gives it the memory blocks it needs,
    \item \textit{root\_partition\_delete} deletes the child partition,
    \item \textit{root\_memory\_cleanup} cleans the memory using interface given by Pip-MPU.
\end{itemize}

These four functions are called in order, and the fifth code part is used between the calls to root\_partition\_create and 
root\_partition\_delete. The purpose of that fifth part is to yield three times to the child partition, in order to let it execute 
completely its program. If we do not take into account the error codes that can be returned by the four functions, the main code of 
the function start is as follows:

\begin{lstlisting}[language=C]
root_memory_init(interface);
root_partition_create();

size_t i;
for (i = 0; i < 3; i++) {
    Pip_yield(child_pd_id, 0, 0, 1, 1);
}

root_partition_delete();
root_memory_cleanup(interface);
\end{lstlisting}

Now let's talk a bit about what the four functions do.

\marginlabel{Set up the memory}

Setting up the memory for the partitions and cleaning it can be tedious, so we will not detail it here. The idea is that, before 
calling the root partition, Pip initializes everything needed. Once Pip's initialization is complete, it has reserved two memory 
blocks, one in the RAM, the other in the flash memory. These two blocks are already guaranteed to be isolated from the rest of the 
system. Some parts of these blocks are already used and should not be touched by the user:
\begin{itemize}
    \item the end of the memory block in the RAM is reserved for Pip,
    \item the beginning of the memory block in the flash memory is reserved for Pip,
    \item the beginning of memory block in the RAM ($R0$ in the table below) serves as the stack of the root partition.
\end{itemize}

\begin{center}
\begin{tabular}{l|c|c|c|}
    \cline{2-4}
    RAM:\hspace{0.5cm} & \hspace{0.5cm}$R0$\hspace{0.5cm} & \hspace{0.5cm}$R1$\hspace{0.5cm} & \hspace{0.3cm}Pip\hspace{0.3cm}\hspace{0.3cm}\\
    \cline{2-4}
\end{tabular}

\begin{tabular}{l|c|c|}
    \cline{2-3}
    flash:\hspace{0.64cm} & \hspace{0.4cm}Pip\hspace{0.4cm} & \hspace{1.35cm}$R2$\hspace{1.35cm}\hspace{1cm}\\
    \cline{2-3}
\end{tabular}
\end{center}

The first step, performed by root\_memory\_init, is to cut $R1$ and $R2$ into isolated blocks and to initialize those blocks to 
prepare for the creation of the child partition. %More details about that will be given in \cref{subsec:details}, but the state 
%of the memory after this initialization is as follows:
The state of the memory after this initialization looks like this:

\begin{center}
\hspace{-2cm}\begin{tabular}{l|c|c|c|c|c|c|c|c|}
    \cline{2-9}
    \multirow{2}*{RAM:\hspace{0.1cm}} & \multirow{2}*{\hspace{0.05cm}$R0$\hspace{0.05cm}} & \multirow{2}*{kernel} & child's & child's & child's stack & child's & \hspace{0.3cm} & \multirow{2}*{\hspace{0.05cm}Pip\hspace{0.05cm}}\\
     & & & descriptor & kernel & + VIDT & context & & \\
    \cline{2-9}
\end{tabular}

\begin{tabular}{l|c|c|c|}
    \cline{2-4}
    flash:\hspace{0.3cm} & \hspace{0.3cm}Pip\hspace{0.3cm} & \hspace{0.2cm}child's text\hspace{0.2cm} & \hspace{3cm}\hspace{1cm}\\
    \cline{2-4}
\end{tabular}
\end{center}

\marginlabel{Create the child partition}

When the previous step is done, the memory is cut into isolated blocks, but all of these blocks still belong to the root partition. 
At this point, we need to create the child partition, give it the blocks it needs, and initialize them. We will not cover the 
initialization here, only an idea of what root\_partition\_create does.

The first important part here is the creation of the child partition. To that end, we can use  the function \textit{Pip\_createPartition}. 
Formally, that function takes only one parameter, a pointer blockLocalId that indicates the index of the child's partition descriptor. 
That index is obtained while allocating the memory space necessary for the child partition, as the return value of the function 
\textit{Pip\_cutMemoryBlock}. Each time we cut a block, we store that ID, and we can use it to refer to the next block. Thus, to 
create a child partition assigned to the block associated to the ID \textit{child\_pd\_id}, we need to run the following line:

\begin{lstlisting}[language=C]
Pip_createPartition(child_pd_id);
\end{lstlisting}

However, this adds only the partition descriptor block to the child partition. The other blocks must be separately given to it, using the 
function \textit{Pip\_addMemoryBlock}. That function takes the ID of a child of the current partition, and shares a block with it, with the 
rights given as parameters. Formally, the function has five parameters:
\begin{itemize}
    \item a pointer childPartDescBlockLocalId, which gives the ID of the child we want to give a block to,
    \item a pointer blockToShareLocalId, which gives the ID of the block to share,
    \item an integer r that should differ from 0 if the child should be able to read the contents of the block,
    \item an integer w that should differ from 0 if the child should be able to write in the block,
    \item an integer e that should differ from 0 if the child should be able to execute the contents of the block.
\end{itemize}

Note that the rights a partition can give using this function do not exceed the ones it has. For example, if a partition does not have the 
right to write in a block, it cannot give that right to any of its children.

%TODO give the code using addMemoryBlock?

Once all the blocks have been added, we initialize them and the corresponding MPU memory. Then, we can yield to the child.

\marginlabel{The main code}

Once all is correctly initialized, the root partition yields three times to its child, to let it execute completly its code. To do so, we 
use the following loop:
\begin{lstlisting}[language=C]
for (i = 0; i < 3; i++) {
    Pip_yield(child_pd_id, 0, 0, 1, 1);
}
\end{lstlisting}

\marginlabel{Delete the child partition}

Once the child partition's execution is done, we want to clean the memory before looping indefinitely. To do so, we call 
root\_partition\_delete, which begins by uninitializing the child's blocks, and removing them from the child partition using 
\textit{Pip\_removeMemoryBlock}. That function takes only one parameter, a pointer blockToRemoveLocalId that gives the ID (inside 
the parent) of the block to remove from a child of the current partition, and removes it. The block still exists and is isolated 
from the others, but the child partition has no longer access to it. Once the uninitialization and the removal of the blocks are 
done, the child partition is empty, save for its partition descriptor. At this point, we can delete the child partition using the 
function \textit{Pip\_deletePartition}. It takes only one parameter, a pointer childPartDescBlockLocalId, which indicates the ID 
of the child partition. This deletes the child partition, leaving all the blocks inside it isolated from one another. So, we use 
the following line:
\begin{lstlisting}[language=C]
Pip_deletePartition(child_pd_id)
\end{lstlisting}

Now, what's left to do is to merge the blocks to get the memory back in its initial state. The blocks that belonged to the child 
partition are handled by root\_partition\_delete, while the ones belonging only to the root are handled by root\_memory\_cleanup.

%\subsection{Managing the memory with Pip}\label{subsec:details}

%This subsection gives detailled explanations about the auxilliary functions mentionned in \cref{subsec:overview}.

%TODO someday

\section{Part IV: Flash and run} \label{fourth}
At this step, you have a binary file including the Pip-MPU kernel and the two partitions of the example project.
The final step is to flash the binary on the target board.

There are many options to flash the executable binary on the target, locally on a board if at disposal or even remotely.
The tutorial covers the flash on a local DWM1001-DEV board using \textit{openocd}.

Another option could be conducting an experiment on a board from the \href{https://www.iot-lab.info/}{FIT IoT-Lab platform}.

\subsection{Install required tools}
\begin{lstlisting}[style=BashStyle]
> sudo apt install openocd
\end{lstlisting}

\subsection{Run openocd}
After the installation of openocd, run it in a Terminal (Terminal 1):

\begin{lstlisting}[style=BashStyle]
> openocd -c 'source [find interface/jlink.cfg]' -c 'transport select swd' -c 'source [find target/nrf52.cfg]' -c 'tcl_port 0' -c 'telnet_port 0' -c 'gdb_port 3333' -c 'init' -c 'targets' -c 'reset halt' -c 'flash write_image erase "pip+root.bin" 0' -c 'verify_image "pip+root.bin" 0' -c 'reset run' -c 'halt'
\end{lstlisting}
Note: if the file \textit{nrf52.cfg} is not found, check if it is present using the absolute path 
\texttt{/usr/share/openocd/scripts/target/nrf52.cfg}. If it is present, run the command again by replacing 
\texttt{[find .../nrf52.cfg]} by the absolute path.

\subsection{Flash}
\textbf{In another terminal (Terminal 2)}, run \texttt{gdb-multiarch} in the main repository to connect to the board and load the binary (also works with \texttt{arm-none-eabi-gdb}).

\begin{lstlisting}[style=BashStyle]
> pipcore-mpu$ gdb-multiarch
(gdb) set architecture arm
(gdb) target extended-remote :3333
(gdb) monitor halt
(gdb) monitor reset
\end{lstlisting}

At that point, the board is flashed and the binary ready to run!

\subsection{See outputs}
The project has been configured in section \ref{buildpip} with the \textit{--debugging-mode=semihosting} option.

In order to view the \textit{semihosting} outputs, you need to enable it.

In the \texttt{gdb-multiarch} terminal (Terminal 2):
\begin{lstlisting}[style=BashStyle]
(gdb) monitor arm semihosting enable
(gdb) continue
\end{lstlisting}

\subsubsection{Option 1 : openocd}
The outputs should be alredy visible in Terminal 1.

\subsubsection{Option 2: Segger}
You need to open a telnet to see the outputs.

\textbf{In another terminal} (Terminal 3):
\begin{lstlisting}[style=BashStyle]
> pipcore-mpu$ telnet localhost 4444
\end{lstlisting}

\section{What's next}
Congratulations, you reached the end of the tutorial 101 on Pip-MPU!
Now you know how to build the Pip-MPU kernel, how to build partitions and link them with the kernel in a single executable file and finally flash and run the firmware on the DWM1001-DEV board.

You can expand your experience on the Pip-MPU kernel by taking a look at the source code and for example change the partition task primitives.

And enjoy the Pip-MPU adventure!



The Pip development team is at disposal for any enquiries regarding this tutorial.


\section{Credits}
This document roots from the Pip tutorial for the x86 architecture.

\end{document}
