%%%%%%%%%%%%%%%%%%%%%%%%%%%%%%%%%%%%%%%%%%%%%%%%%%%%%%%%%%%%%%%%%%%%%%%%%%%%%%%
%  © Université de Lille, The Pip Development Team (2015-2022)                %
%                                                                             %
%  This software is a computer program whose purpose is to run a minimal,     %
%  hypervisor relying on proven properties such as memory isolation.          %
%                                                                             %
%  This software is governed by the CeCILL license under French law and       %
%  abiding by the rules of distribution of free software.  You can  use,      %
%  modify and/ or redistribute the software under the terms of the CeCILL     %
%  license as circulated by CEA, CNRS and INRIA at the following URL          %
%  "http://www.cecill.info".                                                  %
%                                                                             %
%  As a counterpart to the access to the source code and  rights to copy,     %
%  modify and redistribute granted by the license, users are provided only    %
%  with a limited warranty  and the software's author,  the holder of the     %
%  economic rights,  and the successive licensors  have only  limited         %
%  liability.                                                                 %
%                                                                             %
%  In this respect, the user's attention is drawn to the risks associated     %
%  with loading,  using,  modifying and/or developing or reproducing the      %
%  software by the user in light of its specific status of free software,     %
%  that may mean  that it is complicated to manipulate,  and  that  also      %
%  therefore means  that it is reserved for developers  and  experienced      %
%  professionals having in-depth computer knowledge. Users are therefore      %
%  encouraged to load and test the software's suitability as regards their    %
%  requirements in conditions enabling the security of their systems and/or   %
%  data to be ensured and,  more generally, to use and operate it in the      %
%  same conditions as regards security.                                       %
%                                                                             %
%  The fact that you are presently reading this means that you have had       %
%  knowledge of the CeCILL license and that you accept its terms.             %
%%%%%%%%%%%%%%%%%%%%%%%%%%%%%%%%%%%%%%%%%%%%%%%%%%%%%%%%%%%%%%%%%%%%%%%%%%%%%%%

\documentclass[10pt,a4paper,titlepage]{refart}
\usepackage[utf8x]{inputenc}
\usepackage{ucs}
\usepackage[english]{babel}
\usepackage{amsmath}
\usepackage{amsfonts}
\usepackage{amssymb}
\usepackage{listings}
\usepackage{xcolor}
\usepackage{hyperref}

\title{Get Started with Pip}

\definecolor{commentcolor}{rgb}{0,0.6,0}
\definecolor{numbercolor}{rgb}{0.5,0.5,0.5}
\definecolor{stringcolor}{rgb}{0.58,0,0.82}
\definecolor{backgroundcolour}{rgb}{0.95,0.95,0.92}

\lstset {
    backgroundcolor=\color{backgroundcolour},
    basicstyle=\footnotesize,
    breakatwhitespace=false,
    breaklines=true,
    keepspaces=true,
    showspaces=false,
    showstringspaces=false,
    showtabs=false,
    tabsize=4,
    frame=single
}

\iffalse
\lstset{
basicstyle=\ttfamily,
frame=single,
breaklines=true,
showstringspaces=false,
}
\fi

\lstdefinestyle{BashStyle} {
    commentstyle=\color{commentcolor},
    keywordstyle=\color{black},
    stringstyle=\color{stringcolor},
    language=bash
}

\lstdefinestyle{CStyle} {
    emph={uint32_t},
    emphstyle={\color{magenta}},
    commentstyle=\color{commentcolor},
    keywordstyle=\color{magenta},
    stringstyle=\color{stringcolor},
    language=C
}

\hypersetup{
    colorlinks,
    citecolor=black,
    filecolor=black,
    linkcolor=black,
    urlcolor=black
}

\author{Nicolas Dejon - Damien Amara}
\title{Pip-MPU 101 -  Tutorial \\ Getting started on ARM Cortex-M \\ Example with the scheduler partition on the DWM1001-DEV (Arm Cortex-M4) board\\}

\begin{document}
\maketitle
\tableofcontents
%\addcontentsline{toc}{section}{Listings}
%\lstlistoflistings
\pagebreak

This tutorial sets up a building environment for Pip-MPU and gets you through the launch of the \href{(https://gitlab.univ-lille.fr/2xs/pip/pip-mpu-scheduler)}{scheduler partition example}.

It assumes Ubuntu 22.04 LTS users for the commands, but Pip-MPU is known to build on other Linux distributions as well using the appropriate package manager.

The tutorial targets the \href{https://www.qorvo.com/products/d/da007952}{DWM1001-DEV board} by Qorvo which relies on the \href{https://www.nordicsemi.com/products/nrf52832}{nRF52832 chip} (ARM Cortex-M4) by Nordic Semiconductor.

The tutorial is split into three parts.
The first part \ref{first} sets up a working build environment and builds the Pip-MPU kernel.
The second part \ref{second} builds a partition and links it with the kernel in an executable file.
The thrid part \ref{third} addresses the flash and run of the executable on a real, local or remote, board.

\section{Background}

\subsection{Kernel structure}

The kernel is divided into three parts.

\begin{itemize}
    \item HAL: The Hardware Abstraction Layer is used to provide small functions to manipulate the MPU
    \item Core: The logic of Pip-MPU (kernel services)
    \item Boot: The bootstrap code that initializes required hardware and then boots Pip-MPU
\end{itemize}

\subsection{Executable workflow}
The Pip-MPU kernel and its partitions are compiled separately.

Concerning Pip-MPU, Pip-MPU's services (Core folder) are written in Gallina, which is automatically translated in the C language by the Digger tool.
HAL and Boot are written in C/ASM.
Pip-MPU's \texttt{Makefile} includes Digger's translation into C, before compiling the whole as any C project.

Concerning the partitions, they are written in C/ASM.
They have their own \texttt{Makefile} and compiles as any other project.

Hence, in order to get an executable file, a script links together the Pip-MPU kernel and the partitions compiled separately to form a unique executable.
You will get through all these steps in the following.

\section{Part I: Build the kernel} \label{first}

\subsection{Build environment}
In order to create your own partitions on top of Pip-MPU, you need an appropriate development environment, relying on several tools.
These tools are needed for cross-compiling bare-metal C for ARM Cortex-M chips.

\index{gcc}\marginlabel{GNU C Cross Compiler:}
arm-none-eabi-gcc is the only C compiler known to compile Pip-MPU correctly.
CLANG, for example, is not supported. To that end, you need a version of GCC capable of producing 32bits ELF binaries.

The last working version is v12.1.0.

\index{gdb-multiarch} \marginlabel{GNU Debugger multiarch:}
The GNU Debugger allows you to debug a partition while it is executed on the
top of Pip. This is very useful during the development process. That's the
reason why GDB is not mandatory but highly recommended.


\index{stack}\marginlabel{Haskell Stack:}
Pip-MPU uses Digger, a home-made extractor to convert Coq code into C code.
In order to compile this Extractor, which is written in Haskell, we use the Stack tool to download and install automatically the required GHC and libraries.

The last working versions of stack is v2.7.5 and of Digger is v0.3.0.0.

\index{coq}\marginlabel{Coq Proof Assistant:}
Pip-MPU's source code and formal proof of its memory isolation properties are written
using the Coq proof assistant. In order to compile Coq files and
generate the required intermediate files for the kernel to build, you will need the 8.13.1 version of Coq.

The last working version of Coq is 8.13.1.

\index{opam} \marginlabel{OCaml Package Manager:}
Opam is the package manager for the OCaml programming language, the language in
which Coq is implemented. This is the proper way to install and pin the Coq
Proof Assistant to a specific version.

\index{gnu}\marginlabel{GNU Toolchain:}
GNU Sed and GNU Make are required to perform the compilation.

The last working versions of sed is v4.8 and of make is v4.3.

\index{doxygen}\marginlabel{Doxygen:}
Pip-MPU's documentation is generated through CoqDoc (included with Coq) for the Coq part, and Doxygen for the C part.
The documentation is not mandatory to compile Pip-MPU, but it is highly required that you compile it and keep it somewhere safe so you always have some reference to read if you need some information about Pip-MPU's internals.

The last working version of doxygen is v1.9.3.

\subsection{Install required packages}

Update the apt package index before installing the packages:

\begin{lstlisting}[style=BashStyle]
> sudo apt update
> sudo apt install git make opam gdb-multiarch texlive-latex-base doxygen haskell-stack
\end{lstlisting}

\subsection{Get the source}
\index{Pipsrc}\marginlabel{Pip-MPU source:}
Once your development environment is ready, open a terminal emulator and clone the Pip-MPU repository at \texttt{https://github.com/2xs/pipcore-mpu}.
\textit{Warning:} replace by this repository's name.

\begin{lstlisting}[style=BashStyle]
> git clone https://gitlab.univ-lille.fr/2xs/pip/pipcore-mpu.git
> cd pipcore-mpu
\end{lstlisting}

Get Pip-MPU's version compatible with the scheduler example.

\begin{lstlisting}[style=BashStyle]
> git checkout 35aa9a72047541b3c4a6174c715cf64c25a7b4f4
\end{lstlisting}

This repository contains Pip-MPU's kernel, proof and documentation. Still,
it is not ready to compile yet, as it provides no partition to run.
This is covered in the next section "\textit{Scheduler tutorial}".

\index{diggersrc}\marginlabel{Set up Digger:}
As said previously, we use an extractor to convert Coq code into C code.
The source is included in Pip-MPU's main repository as a submodule.
\begin{lstlisting}[style=BashStyle]
> pipcore-mpu$ git submodule init
# checkout to expected version
> pipcore-mpu$ git submodule update
# build Digger, these could take some minutes
> pipcore-mpu$ make -C tools/digger
\end{lstlisting}
The compilation might complain about a missing GHC version : type in the asked command to install the required GHC version (usually "\texttt{stack setup}"), and type in \texttt{make} again to grab the required libraries and build the extractor.

\index{coqcsrc}\marginlabel{Set up the Coq 8.13.1 environment:}
We use opam to install Coq tools required to extract the code base.

\begin{lstlisting}[style=BashStyle]
> opam switch create coq8.13.1 4.10.2
> opam init
> eval $(opam env)
# pin Coq version to 8.13.1, these could take some minutes
> opam pin add coq 8.13.1
# test installation, the command must end successfully
> coqc -v
\end{lstlisting}

\subsection{Build Pip-MPU} \label{buildpip}
Before doing anything else, you need to configure the partition building toolchain for the dwm1001 board.
The optional argument \textit{--debugging-mode=semihosting} enables semihosting outputs.
Then, compile the kernel.
% TODO: remove --no-command-check
Note that the partitions use the same building toolchain as Pip-MPU's.

\begin{lstlisting}[style=BashStyle]
# configure the building toolchain
> pipcore-mpu$ ./configure.sh --architecture=dwm1001 --debugging-mode=semihosting --no-command-check
# compile the kernel
> pipcore-mpu$ make all
# check Pip-MPU has successfully compiled:
> pipcore-mpu$ ls | grep pip.elf
\end{lstlisting}

If the last instruction returns \textit{pip.elf} file as output, congratulations, you successfully compiled Pip-MPU!

\section{Part II: Link the kernel with the scheduler partition example} \label{second}
Now that Pip-MPU is built, we will launch a partition based on the \href{(https://gitlab.univ-lille.fr/2xs/pip/pip-mpu-scheduler)}{scheduler partition example}
For that, you must compile the partition separately and package it with the kernel in order to get the final binary.

\subsection{Download the partition source}
Clone the example in another directory \textit{"scheduler partition dir"}, outside the Pip-MPU kernel.

\begin{lstlisting}[style=BashStyle]
> cd <another directory>
> git clone https://gitlab.univ-lille.fr/2xs/pip/pip-mpu-scheduler.git
> cd pip-mpu-scheduler
\end{lstlisting}

\subsection{Compile the partition}
Partitions inherit Pip-MPU's toolchain configuration (set up at \ref{buildpip}).
It is then enough to invoke the \texttt{make} command.

\begin{lstlisting}[style=BashStyle]
> pip-mpu-scheduler$ make
\end{lstlisting}

\subsection{Link the partition with the kernel}
Return to the kernel's directory and link Pip-MPU's binary with the partition's binary.

\begin{lstlisting}[style=BashStyle]
> pipcore-mpu$ ./root-partition-linker.sh pip.bin <another directory>/pip-mpu-scheduler/pip-mpu-scheduler.bin
\end{lstlisting}

The script produces the final binary.

\section{Part III: Flash and run} \label{third}
At this step, you have a binary file including the kernel and a partition (e.g. the sheduler partition).
The final step is to flash the binary on the target board.

There are many options to flash the executable binary on the target, locally on a board if at disposal or even remotely.
The tutorial covers the flash on a local DWM1001-DEV board using \textit{openocd} (option 1, preferred)) or SEGGER's tools (option 2).
The required tools differ for the two options but the flash and control of the board is the same via \textit{gdb-multiarch}.

Another option could be conducting an experiment on a board from the \href{https://www.iot-lab.info/}{FIT IoT-Lab platform}.

\subsection{Option 1: openocd}

\subsubsection{Install required tools}
\begin{lstlisting}[style=BashStyle]
> sudo apt install openocd
\end{lstlisting}

\subsubsection{Run openocd}
After the installation of openocd, run it in a Terminal (Terminal 1):

\begin{lstlisting}[style=BashStyle]
> openocd -c 'adapter driver jlink; transport select swd; source [find target/nrf52.cfg]' -c init -c "reset init" -c halt
\end{lstlisting}
Note: if the file \textit{nrf52.cfg} is not found, check its presence at \texttt{/usr/share/openocd/scripts/target/nrf52.cfg} and if it is, run the command again by replacing \texttt{[find .../nrf52.cfg]} by the absolute path.


\subsection{Option 2: Segger}
\subsubsection{Install required tools}
Install the J-Link Software and Documentation pack from \href{https://www.segger.com/downloads/jlink/#J-LinkSoftwareAndDocumentationPack}{SEGGER's website}.

%Install the Arm GNU Toolchain from \href{https://developer.arm.com/downloads/-/arm-gnu-toolchain-downloads}{Arm's website}.

\subsubsection{Run JLinkGDBServer}
After the installation of the JLink tools, open JLinkGDBServer in a Terminal (Terminal 1):

\begin{lstlisting}[style=BashStyle]
> pipcore-mpu$ /opt/SEGGER/JLink/JLinkGDBServer -device nRF52832_xxAA -endian little -if SWD -port 3333 -swoport 2332 -telnetport 4444 -vd -ir -localhostonly 1 -strict -timeout 0 -nogui
\end{lstlisting}
This command starts the \texttt{JLinkGDBServer} to connect to the SWD port of nRF52832 chip on the DWM1001 and
\begin{itemize}
	\item specifies the \textit{gdb} port to 3333
	\item specifies the \textit{telnet} port to 4444
	\item ir: initializes the CPU registers
	\item vd: verifies the data after downloading
	\item timeout: sets a timeout of 0 ms to fail directly if no target is connected
	\item strict: checks the settings
\end{itemize}

\subsection{Flash}
\textbf{In another terminal (Terminal 2)}, run \texttt{gdb-multiarch} in the main repository to connect to the board and load the binary (also works with \texttt{arm-none-eabi-gdb}).

\begin{lstlisting}[style=BashStyle]
> pipcore-mpu$ gdb-multiarch
(gdb) set architecture arm
(gdb) target extended-remote :3333
(gdb) file pip.elf
(gdb) load
(gdb) monitor halt
(gdb) monitor reset
\end{lstlisting}

At that point, the board is flashed and the binary ready to run!

\subsection{See outputs}
The project has been configured in section \ref{buildpip} with the \textit{--debugging-mode=semihosting} option.

In order to view the \textit{semihosting} outputs, you need to enable it.

In the \texttt{gdb-multiarch} terminal (Terminal 2):
\begin{lstlisting}[style=BashStyle]
(gdb) monitor semihosting enable
(gdb) continue
\end{lstlisting}

\subsubsection{Option 1 : openocd}
The outputs should be alredy visible in Terminal 1.

\subsubsection{Option 2: Segger}
You need to open a telnet to see the outputs.

\textbf{In another terminal} (Terminal 3):
\begin{lstlisting}[style=BashStyle]
> pipcore-mpu$ telnet localhost 4444
\end{lstlisting}

\section{What's next}
Congratulations, you reached the end of the tutorial 101 on Pip-MPU!
Now you know how to build the Pip-MPU kernel, how to build partitions and link them with the kernel in a single executable file and finally flash and run the firmware on the DWM1001-DEV board.

You can expand your experience on the Pip-MPU kernel by taking a look at the source code and for example change the partition task primitives.

And enjoy the Pip-MPU adventure!



The Pip development team is at disposal for any enquiries regarding this tutorial.


\section{Credits}
This document roots from the Pip tutorial for the x86 architecture.

\end{document}
